\chapter{Introdução à Indexação}

\section{O que é indexação?}

Um índice é uma estrutura de dados que melhora a velocidade das operações de query de dados em uma tabela, ao custo de escritas adicionais e espaço de armazenamento para manter a estrutura do índice. Índices permitem localizar dados rapidamente sem precisar buscar sequencialmente em cada linha de uma tabela.

A maioria dos softwares de banco de dados inclui tecnologia de indexação que permite buscas em tempo sub-linear para melhorar o desempenho, já que a busca linear é ineficiente para grandes bancos de dados.

\cite{databaseindex:wiki}

\section{Implementações de índices}

\subsection{B Tree}

Uma \textbf{B-tree} é uma estrutura de dados em árvore auto-balanceada que mantém dados ordenados e permite buscas, acessos sequenciais, inserções e deleções em tempo logarítmico. A B-tree generaliza a árvore binária de busca, permitindo nós com mais de dois filhos.

É amplamente utilizada em sistemas de arquivos e bancos de dados. É uma estrutura que se beneficia da leitura e escrita em bloco, levando vantagem em um aspecto historicamente relevante, uma vez que o número de operações de I/O (em discos magnéticos) era igualmente relevante para o desempenho quanto o número de operações de comparação.

Foi inventada por Rudolf Bayer e Edward M. McCreight em 1972 (o B não foi explicado por eles).

% Quote
\begin{quotation}
    \it What Rudy (Bayer) likes to say is, the more you think about what the B in B-Tree means, the better you understand B-Trees!
\end{quotation}

Os principais algoritmos associados a B-trees são: busca (\cref{alg:btree_search}) e inserção (\cref{alg:btree_insertion}) (existem variações para a operação de deleção).

São necessárias duas funções auxiliares para a inserção: \textsc{splitChild}, que divide um nó cheio em dois, e \textsc{insertNonFull}, que insere uma chave em um nó não cheio.

\begin{algorithm}
\caption{Algoritmo de busca na B Tree, assumindo que a chave $k$ é o valor a ser buscado e $x$ é o nó onde a busca começa.}
\label{alg:btree_search}
\begin{algorithmic}[1]
\Procedure{BtreeSearch}{$x, k$}
    \State $i \gets 0$
    \While{$i < x.n$ \textbf{and} $k > x.key[i]$}
        \State $i \gets i + 1$
    \EndWhile
    \If{$i < x.n$ \textbf{and} $k = x.key[i]$}
        \State \Return $x$
    \EndIf
    \If{$x.leaf$}
        \State \Return \textbf{None}
    \EndIf
    \State \Return \Call{BtreeSearch}{$x.child[i], k$}
\EndProcedure
\end{algorithmic}
\end{algorithm}

\begin{algorithm}
\caption{Algoritmo de inserção na B Tree, assumindo que a chave $k$ é o valor a ser inserido.}
\label{alg:btree_insertion}
\begin{algorithmic}[1]
\Procedure{BtreeInsert}{$T, k$}
    \State $r \gets T.root$
    \If{$r.n = 2(T.d) - 1$}
        \State $s \gets$ \textbf{new} Node
        \State $T.root \gets s$
        \State $s.child[1] \gets r$
        \State \Call{splitChild}{$s, 1$}
        \State \Call{insertNonFull}{$s, k$}
    \Else
        \State \Call{insertNonFull}{$r, k$}
    \EndIf
\EndProcedure
\end{algorithmic}
\end{algorithm}

\cite{btree:wiki}

\subsection{B+ Tree}

\cite{bptree:wiki}

\printbibliography